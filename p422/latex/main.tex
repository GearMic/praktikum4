%% packages
\documentclass{article}
\usepackage[a4paper, left=2.0cm, right=2.0cm, top=3.5cm]{geometry}
\usepackage[ngerman]{babel}
\usepackage{graphicx}
\usepackage{multicol}
\usepackage{amssymb}
\usepackage{titlesec}
\usepackage{wrapfig}
\usepackage{blindtext}
\usepackage{lipsum}
\usepackage{caption}
\usepackage{listings}
\usepackage{fancyhdr}
\usepackage{nopageno}
\usepackage{authblk}
\usepackage{amsmath} % tons of math stuff
\usepackage{mathtools} % e.g. alignment within matrix
%\usepackage{bm} % provides shorthand for bold in math mode
\usepackage{dsfont} % \mathds makes double stroke digits
\usepackage{esdiff} % provides \diff
%\usepackage[ISO]{diffcoeff}
\usepackage{xcolor}
\usepackage{csquotes} % e.g. provides \enquote
\usepackage[separate-uncertainty=true]{siunitx} % units
\usepackage{xcolor} % colored text
\usepackage{csvsimple}
\usepackage{subcaption}
\usepackage{physics}
\usepackage{hyperref}
\usepackage{nameref}
\hypersetup{colorlinks=true, linkcolor=black, pdfhighlight={/N}}
\usepackage{tcolorbox}
\usepackage{amsthm}




%\fancyhf[]{}

%% custom stuff
% own units
\DeclareSIUnit \VSS {\ensuremath{V_\mathrm{SS}}}
\DeclareSIUnit \VS {\ensuremath{V_\mathrm{S}}}
\DeclareSIUnit \Veff {\ensuremath{V_\mathrm{eff}}}
\DeclareSIUnit \Vpp {\ensuremath{V_\mathrm{pp}}}
\DeclareSIUnit \Vp {\ensuremath{V_\mathrm{p}}}
\DeclareSIUnit \VRMS {\ensuremath{V_\mathrm{RMS}}}
\DeclareSIUnit \ASS {\ensuremath{A_\mathrm{SS}}}
\DeclareSIUnit \AS {\ensuremath{A_\mathrm{S}}}
\DeclareSIUnit \Aeff {\ensuremath{A_\mathrm{eff}}}
\DeclareSIUnit \App {\ensuremath{A_\mathrm{pp}}}
\DeclareSIUnit \Ap {\ensuremath{A_\mathrm{p}}}
\DeclareSIUnit \ARMS {\ensuremath{A_\mathrm{RMS}}}

% change subsection numbering to capital letters
\newcommand{\subsectionAlph}{ \renewcommand{\thesubsection}{\arabic{section}.\Alph{subsection}} }
% change subsection numbering to lowercase letters
\newcommand{\subsectionalph}{ \renewcommand{\thesubsection}{\arabic{section}.\alph{subsection}} }
% change subsubsection numbering to lowercase letters
\newcommand{\subsubsectionalph}{ \renewcommand{\thesubsubsection}{\arabic{section}.\arabic{subsection}.\alph{subsubsection}} }
% own fig. that works with multicols
\newenvironment{Figure}
  {\par\medskip\noindent\minipage{\linewidth}}
  {\endminipage\par\medskip}
\newcommand*{\inputPath}{./plot} % prepend this command to the argument of all input commands
\graphicspath{ {./images/}{./figure/}{../plot/} }
% own enviroment for definitions
\newenvironment{definition}[1]
{\begin{quote} \noindent \textbf{\textit{#1\ifx&#1& \else : \fi}} \itshape}
{\end{quote}}


% own commands
% \newcommand{\rarr}{$\to\,$} %A$\,\to\,$B
\newcommand{\defc}{black}
\newcommand{\colorT}[2][blue]{\color{#1}{#2}\color{\defc}}
\newcommand{\redq}{\color{red}(?)\color{\defc}}
\newcommand{\question}[1]{\colorT[purple]{\textbf{(#1)}}}
\newcommand{\todo}[1]{\colorT[red]{\textbf{(#1)}}}
\newcommand{\mr}{\mathrm}

%% preparation
\begin{titlepage}
    \title{Praktikum Atome, Moleküle, kondensierte Materie \\ Versuch 401}
    \author[1]{Carlos Pascua\thanks{s87cpasc@uni-bonn.de}}
    \author[1]{Michael Vogt\thanks{s65mvogt@uni-bonn.de}}
    \affil[1]{Uni Bonn}
    %\date{\today}
\end{titlepage}


%% document
\begin{document}

\pagenumbering{gobble}
\maketitle
\tableofcontents
\newpage
\pagenumbering{arabic}

\pagestyle{fancy}
\fancyhead[R]{\thepage}
\fancyhead[L]{\leftmark}

\begin{multicols}{2}

\section*{Einleitung}


\section{Durchführung \& Auswertung}
Ein Schema der Struktur von HOPG ist in Abb. \ref{fig:hopg-structure} gezeigt. Bei Betrachtung der Schichten senkrecht von oben,
wie hier mit dem STM, gibt es drei relevante Formen sichtbarer Strukturen:
\begin{enumerate}
  \item Ein Atom in der obersten Schicht und keins in der Schicht darunter
  \item Ein Atom in der obersten Schicht und ein Atom in der Schicht darunter
  \item Kein Atom in der obersten Schicht, aber ein Atom in der Schicht darunter
\end{enumerate}
Diese drei Strukturen führen an der Oberfläche zu drei unterschiedlich hohen Elektronendichten,
die im STM-Bild in Form von drei verschiedenen Helligkeitsstufen (Hell, Halb-Hell, Dunkel) sichtbar werden.
In einem optimalen STM-Bild sollten Punkte der gleichen Helligkeitsstufe jeweils in
einem Gitter gleichseitiger Dreiecke mit Kantenlänge $a$ (Gitterkonstante; siehe Abb. \ref{fig:hopg-structure}) angeordnet sein.

In Abb. \ref{fig:hopg-medium} sind mehrere Linien eingezeichnet, um aus dem Abstand der dunklen Punkte die Gitterkonstante $a$
zu bestimmen. Dabei wurden möglichst viele Gitterabstände überspannt, um den Fehler zu minimieren.
Die Maße der Linien sind in Tab. \ref{tab:hopg-data} eingetragen. $a$ ergibt sich durch $a=\frac{R}{n}$, wobei $n$
der Anzahl der von der Linie überspannten Gitterabständen entspricht;
$\Delta a$ wurde mithilfe von Gauß'scher Fehlerfortpflanzung berechnet.

Der Literaturwert ist $a=\SI{0.246}{\nm}$, wovon der aus Linien $1$ und $3$ bestimmte $a$-Wert um ca. $\SI{4}{\percent}$
nach oben abweicht. Die Werte aus Linien $2$ und $4$ hingegen sind um mehr als \SI{10}{\percent} niedriger.
Dies deutet darauf hin, dass die Piezos für die x- und y- Richtung nicht optimal kalibriert sind.
Linien $1$ und $3$, die bessere $a$-Werte liefern, verlaufen vorwiegend horizontal, also scheint die Kalibration des x-Piezos
eher zu stimmen. Linien $2$ und $4$, die eine größere Ausdehnung in der Vertikalen haben, sind deutlich zu kurz,
also muss der $y-Piezo$ so kalibriert sein, dass er sich zu weit bewegt und somit Abstände unterschätzt werden.

Dass die Kalibration in y-Richtung stärker abweicht, als in x-Richtung,
führt zu einer Verzerrung des Bilds, welche sich auch in den Winkeln zwischen Linien erkennen lässt: Die Differenz der
Winkel $\phi$ der Linien $1$ und $2$ ist \SI{66.0\pm 4.2}{\degree} und zwischen Linien $3$ und $4$ \SI{67.3\pm 4.2}{\degree},
was um ca. \SI{10}{\percent} höher ist, als der Winkel \SI{66}{\degree}, der bei einem gleichseitigen Dreieck eigentlich zu
erwarten wäre.
% Aus Linien $1$ und $3$ erhalten wir $a=0.256$ und

\clearpage

\clearpage
\section{Fazit}


\clearpage
\begin{thebibliography}{9}

\bibitem{Anleitung}
\textit{Physikalisches Praktikum Teil IV -- Versuchsbeschreibungen}, Universität Bonn, 10.10.2024


\end{thebibliography}

\end{multicols}

\end{document}

