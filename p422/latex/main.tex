%% packages
\documentclass{article}
\usepackage[a4paper, left=2.0cm, right=2.0cm, top=3.5cm]{geometry}
\usepackage[ngerman]{babel}
\usepackage{graphicx}
\usepackage{multicol}
\usepackage{amssymb}
\usepackage{titlesec}
\usepackage{wrapfig}
\usepackage{blindtext}
\usepackage{lipsum}
\usepackage{caption}
\usepackage{listings}
\usepackage{fancyhdr}
\usepackage{nopageno}
\usepackage{authblk}
\usepackage{amsmath} % tons of math stuff
\usepackage{mathtools} % e.g. alignment within matrix
%\usepackage{bm} % provides shorthand for bold in math mode
\usepackage{dsfont} % \mathds makes double stroke digits
\usepackage{esdiff} % provides \diff
%\usepackage[ISO]{diffcoeff}
\usepackage{xcolor}
\usepackage{csquotes} % e.g. provides \enquote
\usepackage[separate-uncertainty=true]{siunitx} % units
\usepackage{xcolor} % colored text
\usepackage[l3]{csvsimple}
\usepackage{subcaption}
\usepackage{physics}
\usepackage{hyperref}
\usepackage{nameref}
\hypersetup{colorlinks=true, linkcolor=black, pdfhighlight={/N}}
\usepackage{tcolorbox}
\usepackage{amsthm}
\usepackage{gensymb} % add \degree in math mode?
\usepackage{newunicodechar} % define custom unicode characters
\usepackage{booktabs}
\usepackage{subcaption}

% \sisetup{
%   scientific-notation = auto,  % Automatically use scientific notation for large/small numbers
%   output-exponent-marker = \text{e}  % (optional) for formatting the exponent symbol
% }



%\fancyhf[]{}

%% custom stuff
% own units
\DeclareSIUnit \VSS {\ensuremath{V_\mathrm{SS}}}
\DeclareSIUnit \VS {\ensuremath{V_\mathrm{S}}}
\DeclareSIUnit \Veff {\ensuremath{V_\mathrm{eff}}}
\DeclareSIUnit \Vpp {\ensuremath{V_\mathrm{pp}}}
\DeclareSIUnit \Vp {\ensuremath{V_\mathrm{p}}}
\DeclareSIUnit \VRMS {\ensuremath{V_\mathrm{RMS}}}
\DeclareSIUnit \ASS {\ensuremath{A_\mathrm{SS}}}
\DeclareSIUnit \AS {\ensuremath{A_\mathrm{S}}}
\DeclareSIUnit \Aeff {\ensuremath{A_\mathrm{eff}}}
\DeclareSIUnit \App {\ensuremath{A_\mathrm{pp}}}
\DeclareSIUnit \Ap {\ensuremath{A_\mathrm{p}}}
\DeclareSIUnit \ARMS {\ensuremath{A_\mathrm{RMS}}}

% change subsection numbering to capital letters
\newcommand{\subsectionAlph}{ \renewcommand{\thesubsection}{\arabic{section}.\Alph{subsection}} }
% change subsection numbering to lowercase letters
\newcommand{\subsectionalph}{ \renewcommand{\thesubsection}{\arabic{section}.\alph{subsection}} }
% change subsubsection numbering to lowercase letters
\newcommand{\subsubsectionalph}{ \renewcommand{\thesubsubsection}{\arabic{section}.\arabic{subsection}.\alph{subsubsection}} }
% own fig. that works with multicols
\newenvironment{Figure}
  {\par\medskip\noindent\minipage{\linewidth}}
  {\endminipage\par\medskip}
\newcommand*{\inputPath}{./plot} % prepend this command to the argument of all input commands
\graphicspath{ {./images/}{./figure/}{../plot/}{../../plot/}{../../latex/assets/}{./assets/} }
% own enviroment for definitions
\newenvironment{definition}[1]
{\begin{quote} \noindent \textbf{\textit{#1\ifx&#1& \else : \fi}} \itshape}
{\end{quote}}

\newunicodechar{°}{\degree}


% own commands
% \newcommand{\rarr}{$\to\,$} %A$\,\to\,$B
\newcommand{\defc}{black}
\newcommand{\colorT}[2][blue]{\color{#1}{#2}\color{\defc}}
\newcommand{\redq}{\color{red}(?)\color{\defc}}
\newcommand{\question}[1]{\colorT[purple]{\textbf{(#1)}}}
\newcommand{\todo}[1]{\colorT[red]{\textbf{(#1)}}}
\newcommand{\mr}{\mathrm}

%% preparation
\begin{titlepage}
    \title{Praktikum Atome, Moleküle, kondensierte Materie \\ Versuch 422: Rastertunnelmikroskopie}
    \author[1]{Carlos Pascua\thanks{s87cpasc@uni-bonn.de}}
    \author[1]{Michael Vogt\thanks{s65mvogt@uni-bonn.de}}
    \affil[1]{Uni Bonn}
    %\date{\today}
\end{titlepage}


%% document
\begin{document}

\pagenumbering{gobble}
\maketitle
\tableofcontents
\newpage
\pagenumbering{arabic}

\pagestyle{fancy}
\fancyhead[R]{\thepage}
\fancyhead[L]{\leftmark}

\section*{Einleitung}
Es werden Goldkugel-Proben sowie HOPG auf sehr kleinen Skalen mithilfe eines Rastertunnelmikroskops untersucht.
Das HOPG (Highly Ordered Pyrolithic Graphite \cite{Anleitung}) wird genauer auf seine atomare Struktur untersucht.

\section{Rastertunnelmikroskopie}
Hier soll u.a. die atomare Struktur einer Probe von HOPG (Highly Ordered Pyrolithic Graphite) untersucht werden.
Um atomare Auflösung zu erreichen, wird ein Rastertunnelmikroskop (RTM/STM) eingesetzt.
Dieses ist im Gegensatz zu Mikroskopen, die Welleneffekte verwenden, wie optische oder Elektronenmikroskope,
in seiner Auflösung nicht durch Beugung begrenzt.
Stattdessen wird eine Metallspitze sehr nah an die zu untersuchende Oberfläche 
herangefahren und anhand des stark von dem Abstand Spitze--Probe abhängigen Tunnelstroms auf deren Oberflächenbeschaffenheit geschlossen.

Hier soll u.a. die atomare Struktur einer Probe von HOPG (Highly Ordered Pyrolithic Graphite) untersucht werden.
Um atomare Auflösung zu erreichen, wird ein Rastertunnelmikroskop (RTM/STM) eingesetzt.
Dieses ist im Gegensatz zu Mikroskopen, die Welleneffekte verwenden, wie optische oder Elektronenmikroskope,
in seiner Auflösung nicht durch Beugung begrenzt.
Stattdessen wird eine Metallspitze sehr nah an die zu untersuchende Oberfläche 
herangefahren und anhand des stark von dem Abstand Spitze--Probe abhängigen Tunnelstroms auf deren Oberflächenbeschaffenheit geschlossen.

% Rastertunnelmikroskope (Scanning Tunneling Microscope, RTM \cite{Anleitung}) sind keine Mikroskope im optischen Sinne, sondern tasten
% die abzubildende Oberfläche in kleinen Schritten ab, um Informationen über deren
% Beschaffenheit zu erhalten. Dadurch sind sie, im Gegensatz zu optischen oder
% Elektronenmikroskopen, in ihrer Auflösung nicht durch Beugung begrenzt \todo{zitieren?}.
% Somit kann bei korrekter Handhabung eine atomare Auflösung von Bruchteilen von Nanometern erreicht werden.

% Die Funktion von STMs basiert auf dem Tunneleffekt, durch den Elektronen 
% in der zu untersuchenden Probe die Austrittsarbeit des Materials sowie 
% die Breite des Spaltes zwischen Probe und Spitze überwinden
\subsection*{Tunneleffekt} 
Um Informationen über eine Oberfläche zu erhalten, ohne sie zu berühren, nutzen STMs den
Tunneleffekt, durch den Elektronen von der Probe zur Spitze übergehen können.
Der Tunneleffekt beschreibt quantenmechanisch, wie Teilchen eine Potentialbarriere durchqueren können, ohne über die hierfür notwendige klassische Energie zu verfügen.

Dazu müssen sie in der zu untersuchenden Probe die Austrittsarbeit des Materials 
sowie  die Breite des Spaltes zwischen Probe und Spitze überwinden,
was klassisch nicht möglich wäre. Durch die Quantenmechanik ergibt sich jedoch eine
Tunnelwahrscheinlichkeit, welche exponentiell vom Abstand Probe--Spitze abhängt \cite{stm-paper}.

Probe und Spitze werden elektrisch verbunden. Dadurch gleichen sich die Fermi-Niveaus (oberste belegte Energie-Niveaus) in beiden Materialien aneinander an.
Elektronen können jedoch nur von Probe zu Spitze tunneln, wenn ihr entsprechende Energie-Niveau
in der Spitze frei ist. Zur Verschiebung der Energieniveuas muss also eine Spannung zwischen Spitze und Probe angelegt werden.
Die Tunnelnden Elektronen führen dann zu einem \textit{Tunnelstrom}, welcher verstärkt,
zu Spannung umgewandelt und gemessen wird. Dieser Strom ist stärker, je größer die
anliegende Spannung ist und hängt,
wie zuvor erklärt, stark von der Distanz Spitze--Probe ab. Eine Änderung des Abstands
um $\SI{100}{\nm}$ führt zu einer Änderung des Stroms um einen Faktor $10$ \cite{naio-video}.
Da deshalb nur der Punkt der Spitze, welcher der Probe am nächsten ist, einen signifikanten Beitrag zum Tunnelstrom liefert,
lässt sich die Spitze oft als punktförmig nähern.
Hat die Spitze mehrere Atome nebeneinander mit (fast) gleichem Abstand zur Probe,
ergibt sich das resultierende Bild mathematisch gesehen als Faltung der Formen
von Spitze und Probe \cite{Anleitung}. Dies kann zu mehrfacher Abbildung von 
Merkmalen der Probe führen.

Um eine Probe abzubilden, wird die Spitze in einem Raster über die Oberfläche 
gefahren und für jede Position die Werte des Abstands und des Tunnelstroms aufgenommen.
Der Strom wird dabei durch eine PID-Regelung konstant gehalten (\textit{constant current mode}), wozu bei einer Höhenänderung ggf. die z-Position nachjustiert werden muss. In diesem Modus kann die Struktur der Oberfläche also anhand der z-Positionen
der Spitze erkannt werden.

\subsection*{Theorie}
\subsection*{Tunneleffekt} 
Um Informationen über eine Oberfläche zu erhalten, ohne sie zu berühren, nutzen STMs den
Tunneleffekt, durch den Elektronen von der Probe zur Spitze übergehen können.
Der Tunneleffekt beschreibt quantenmechanisch, wie Teilchen eine Potentialbarriere durchqueren können, ohne über die hierfür notwendige klassische Energie zu verfügen.

Dazu müssen sie in der zu untersuchenden Probe die Austrittsarbeit des Materials 
sowie  die Breite des Spaltes zwischen Probe und Spitze überwinden,
was klassisch nicht möglich wäre. Durch die Quantenmechanik ergibt sich jedoch eine
Tunnelwahrscheinlichkeit, welche exponentiell vom Abstand Probe--Spitze abhängt \cite{stm-paper}.

Probe und Spitze werden elektrisch verbunden. Dadurch gleichen sich die Fermi-Niveaus (oberste belegte Energie-Niveaus) in beiden Materialien aneinander an.
Elektronen können jedoch nur von Probe zu Spitze tunneln, wenn ihr entsprechende Energie-Niveau
in der Spitze frei ist. Zur Verschiebung der Energieniveuas muss also eine Spannung zwischen Spitze und Probe angelegt werden.
Die Tunnelnden Elektronen führen dann zu einem \textit{Tunnelstrom}, welcher verstärkt,
zu Spannung umgewandelt und gemessen wird. Dieser Strom ist stärker, je größer die
anliegende Spannung ist und hängt,
wie zuvor erklärt, stark von der Distanz Spitze--Probe ab. Eine Änderung des Abstands
um $\SI{100}{\nm}$ führt zu einer Änderung des Stroms um einen Faktor $10$ \cite{naio-video}.
Da deshalb nur der Punkt der Spitze, welcher der Probe am nächsten ist, einen signifikanten Beitrag zum Tunnelstrom liefert,
lässt sich die Spitze oft als punktförmig nähern.
Hat die Spitze mehrere Atome nebeneinander mit (fast) gleichem Abstand zur Probe,
ergibt sich das resultierende Bild mathematisch gesehen als Faltung der Formen
von Spitze und Probe \cite{Anleitung}. Dies kann zu mehrfacher Abbildung von 
Merkmalen der Probe führen.

Um eine Probe abzubilden, wird die Spitze in einem Raster über die Oberfläche 
gefahren und für jede Position die Werte des Abstands und des Tunnelstroms aufgenommen.
Der Strom wird dabei durch eine PID-Regelung konstant gehalten (\textit{constant current mode}), wozu bei einer Höhenänderung ggf. die z-Position nachjustiert werden muss. In diesem Modus kann die Struktur der Oberfläche also anhand der z-Positionen
der Spitze erkannt werden.

\subsection*{Theorie}
Betrachten wir ein konstantes Potential \( V \), das im Intervall \( I_0 \) den Wert \( V_0 \) annimmt, so gilt für die Wellenfunktion \( \psi(x) \): 
\begin{align*}
    E \psi(x) &= \left[ - \frac{\hbar^2}{2m} \frac{\partial^2}{\partial x^2} + V_0 \right] \psi(x), \\
    0 &= \left[ \frac{\partial^2}{\partial x^2} + \frac{2m(E - V_0)}{\hbar^2} \right] \psi(x).
\end{align*}

Analysiert man das Skalarprodukt der Aufenthaltswahrscheinlichkeit genauer, so ergibt sich die Proportionalität:
\begin{align*}
    \abs{\psi(x)}^2 \propto \exp\left(-\frac{2}{\hbar} \sqrt{2m (E - V_0)} \cdot x\right).
\end{align*}

\begin{figure}[h!]
    \centering
    \includegraphics[width=0.5\linewidth]{figure/tunneleffekt.png}
    \caption{Schematische Darstellung des Tunneleffekts.}
    \label{fig:tunneleffekt}
\end{figure}

\subsection*{Piezo-Effekt}
Zur präzisen Bewegung der Spitze werden Piezo-Kristalle eingesetzt. 
Der Piezoeffekt beschreibt die Fähigkeit bestimmter Materialien, unter mechanischer Belastung elektrische Potentialunterschiede an ihren Oberflächen zu erzeugen.
Diese Spannungen entstehen durch eine Verschiebung von Ladungsschwerpunkten in der Kristallstruktur des Materials, wenn es gedehnt oder zusammengedrückt wird.
Typische piezoelektrische Materialien sind Quarz, Keramiken und einige Polymere.
\subsection*{Piezo-Effekt}
Zur präzisen Bewegung der Spitze werden Piezo-Kristalle eingesetzt. 
Der Piezoeffekt beschreibt die Fähigkeit bestimmter Materialien, unter mechanischer Belastung elektrische Potentialunterschiede an ihren Oberflächen zu erzeugen.
Diese Spannungen entstehen durch eine Verschiebung von Ladungsschwerpunkten in der Kristallstruktur des Materials, wenn es gedehnt oder zusammengedrückt wird.
Typische piezoelektrische Materialien sind Quarz, Keramiken und einige Polymere.

Umgekehrt bewirkt der inverse Piezoeffekt, dass diese Materialien ihre Form ändern, wenn eine elektrische Spannung angelegt wird.
Dieser Effekt wird in verschiedenen Anwendungen wie Sensoren, Aktoren und Ultraschallgeräten genutzt, da er präzise Bewegungen und Messungen ermöglicht.
Umgekehrt bewirkt der inverse Piezoeffekt, dass diese Materialien ihre Form ändern, wenn eine elektrische Spannung angelegt wird.
Dieser Effekt wird in verschiedenen Anwendungen wie Sensoren, Aktoren und Ultraschallgeräten genutzt, da er präzise Bewegungen und Messungen ermöglicht.

\begin{figure}[h!]
    \centering
    \includegraphics[width=0.5\linewidth]{figure/piezo_effekt.jpg}
    \caption{Piezoeffekt}
    \label{fig:piezoelektrische}
\end{figure}

Zur Annäherung der Probe an die Spitze muss diese zunächst eine relativ 
Zur Annäherung der Probe an die Spitze muss diese zunächst eine relativ 
große Distanz (einige \si{\mm} bis \si{cm}) zurücklegen, wozu ein sogenannter Stick-Slip
Piezo-Motor eingesetzt wird. Dieser bewegt sich zunächst langsam nach vorne, wobei
der aufliegende Probenhalter durch Haftreibung mitbewegt wird. Durch eine schnelle
Rückwärtsbewegung kann der Motor in seinen Ausgangszustand gebracht werden, ohne die
Probe mitzuziehen. Diese Bewegung kann mit hoher Frequenz wiederholt werden, um die Probe anzunähern.
Für die Bewegung der Spitze entlang der Probe (seitwärts in $x$- und $y$-Richtung und
vorwärts in $z$-Richtung) müssen schließlich nur sehr kleine Distanzen zurückgelegt
werden (wenige \si{nm} bis wenige hundert \si{\nm}), wozu herkömmliche Piezos ausreichen.

\subsection*{PID-Regler}
\subsection*{PID-Regler}
PID ist eine bestimmte Art von Regelung mit drei Parametern P (\textit{proportional}), I (\textit{integral}) und D (\textit{differential}).
Die Differential-Regelung wird hier i.d.R. weggelassen. Um die Regelung für Änderungen auf kleinen Distanzen auszustellen, kann der P-Parameter auf $0$ und
der I-Parameter auf einen geringen Wert (z.B. $4$ \cite{Anleitung}) gesetzt werden.
So wird die Spannung am z-Piezo bei Höhenänderungen der Oberfläche nicht nachgeregelt,
aber der Strom verändert sich signifikant. Dies ist der \textit{constant height mode},
bei dem Die Oberflächenstruktur aus dem Verlauf des Tunnelstroms folgt.\\

Das Ansteuern der Piezos, die Messung des Tunnelstroms, die PID-Regelung und die Konstruktion der resultierenden Bilder werden
durch einen \textit{STM-Controller} in Kombination mit einer STM-Software bewerkstelligt.

Da die Funktion von STMs auf dem Tunnelstrom basiert, können nur elektrisch leitfähige
Proben vermessen werden. 

% Hierbei wird die Spannung am $z$-Piezo durch drei Terme, welche
% von der Differenz des Soll-Stroms und des tatsächlichen Stroms abhängen, reguliert.
% Der P-Term (\textit{proportional}) ist propo
\subsection*{Gwyddion} 
\subsection*{Gwyddion} 
In \textbf{Gwyddion} können Daten aus STM-Experimenten durch den Import von Dateien, die typische STM-Formate wie \texttt{.sm4} oder \texttt{.ibw} verwenden, analysiert werden. Die Software ermöglicht die Darstellung der Daten als Höhenkarten, sowohl in 2D- als auch in 3D-Ansicht, um die atomaren Strukturen präzise zu visualisieren. Zur Optimierung der Rohdaten stehen Funktionen wie Hintergrundnivellierung, Rauschunterdrückung und Fourier-Transformationen zur Verfügung.

Die Software bietet darüber hinaus Werkzeuge für die quantitative Analyse. Dazu gehören Messungen der Oberflächenrauigkeit, Partikelgrößen und andere strukturelle Parameter. Diese Analysen erlauben eine detaillierte Untersuchung der Topographie und physikalischen Eigenschaften der Probe.

Zusätzlich können in Gwyddion fortgeschrittene Analysen durchgeführt werden, beispielsweise die Extraktion von Linienprofilen oder die Berechnung von statistischen Eigenschaften der Oberfläche. Dank der modularen und benutzerfreundlichen Struktur ist das Programm ein unverzichtbares Werkzeug für die STM-Datenanalyse in der Oberflächenforschung.
Für alle unsere Abbildungen des Versuchs wird dieses Programm verwendet.
Dabei wird immer die Funktion zur Entfernung von Linien und der Untergrundausgleich angewendet.


\subsection*{Aufbau und Durchführung}
Der Versuchaufbau ist in Abb. \ref{fig:aufbau} gezeigt.
\begin{figure}[h!]
    \centering
    \includegraphics[width=0.49\textwidth]{aufbau.png}
    \caption{Versuchaufbau nach der Anleitung \cite{Anleitung}}
    \label{fig:aufbau}
\end{figure}
Es gibt 2 USB-Lupen. Eine wird über dem Positioneniertisch gebracht, um Aufnahmen von verwendeten Spitzen und
Proben zu machen. Der Gegendstand wird positioniert, indem man ein Magnet im x-y-z Positioneniertisch
verbaut. Die andere Lupe wird über dem STM gebracht, um den Anfahrvorgang der Probe zur Spitze zu 
beobachten.

Zusätzlich hat den Versuch einen Box mit dem Inhalt:
\begin{figure}[h!]
    \centering
    \includegraphics[width=0.49\textwidth]{box-inhalt.png}
    \caption{Bauteile, Probe und Pinzitte \cite{Anleitung}}
    \label{fig:aufbau}
\end{figure}
Hier sind alles, was man zum Untersuchen benötigt.
Die Durchführung dieses Versuchs ist leicht nachvollziehbar. Zunächst wird die Probe auf einem
USB-Mikroskop positioniert, um Bilder aufzunehmen, die anschließend für die Beurteilung 
ihres Zustands verwendet werden. Für das Rastertunnelmikroskop (RTM) benötigt man eine 
geeignete Spitze, die in der Regel selbst \textit{gerissen} wird. Mithilfe eines Seitenschneiders 
wird ein Draht mit einem Durchmesser von etwa $0.3 \, \mathrm{mm}$ so abgetrennt, dass 
eine scharfe Spitze entsteht. Für optimale Messergebnisse empfiehlt es sich, den Draht gemäß 
Abbildung \ref{fig:spitze_anl} zuzuschneiden.

\begin{figure}[h!]
    \centering
    \includegraphics[width=0.49\textwidth]{spitzeanleitung.png}
    \caption{Empfohlener Aufbau der RTM-Spitze \cite{easyScan}.}
    \label{fig:spitze_anl}
\end{figure}

Nach der Fertigstellung wird die Spitze unter dem USB-Mikroskop untersucht und fotografiert.
Anschließend wird sie im RTM installiert, während die Proben sicher auf einem Probenhalter
befestigt werden, der sich auf dem Stick-Slip-Antrieb des Geräts befindet. Mithilfe der
RTM-Software kann die Probe kontrolliert näher an die Spitze bewegt werden. Idealerweise
wird der Abstand manuell so weit verringert, dass er nicht mehr sichtbar ist und danach mit der
Funktion \textit{Approach} der endgültige Tunnelkontakt hergestellt. 

\newpage
\section{Goldkugelprobe}
%%\todo{welche Probe ist das?}

Zur Verfügung standen zwei HOPG-Proben sowie zwei Kristallproben, auf die Gold aufgedampft wurde. Während des Aufdampfens bildeten sich kugelartige Strukturen aus, deren Radien bei den beiden Proben unterschiedlich waren. Im Folgenden wird die Probe untersucht, bei der Gold auf Silizium aufgebracht wurde, gekennzeichnet als Probe 3.

Die verwendete STM-Spitze ist in Abb. \ref{fig:spitze-1} dargestellt, während ein Lupenbild der Probe in Abb. \ref{fig:goldkugel-lupe} gezeigt wird. Um den Maßstab zu bestimmen, wurde zunächst ein 400x100-TEM-Netz mit der Lupe fotografiert. Anschließend wurde die Probe auf das Raster gelegt, und die Lupenhöhe so eingestellt, dass die Oberseite der Probe auf derselben Höhe wie das TEM-Netz liegt. Das Netz weist Abstände von $\frac{1}{100}$ Zoll auf, was einer Länge von $\SI{0.254}{\mm}$ pro Rechteck entspricht \cite{meshsize}.

\begin{figure}[h!]
    \centering
    %\includegraphics[width=0.49\textwidth]{Spitze1}
    \caption{STM-Spitze, die zur Untersuchung der Goldprobe verwendet wurde. Der Maßstab ergibt sich aus der Drahtdicke von \SI{0.3}{\mm}. Die Spitze stammt aus einem Behälter mit alten Spitzen, weshalb sie möglicherweise nicht einatomig war.}
    \label{fig:spitze-1}
\end{figure}
\newpage
\begin{figure}[h!]
    \centering
    %\includegraphics[width=0.49\textwidth]{GitterGold}
    \hfill
    %\includegraphics[width=0.49\textwidth]{Gold}
    \caption{Links: TEM-Netz zur Maßstabbestimmung. Rechts: optisches Bild der Goldkugel-Probe.}
    \label{fig:goldkugel-lupe}
\end{figure}

Obwohl die Probe sichtbare Kratzer und Unebenheiten aufwies, wurde sie mit dem STM untersucht, da keine besser erhaltenen Goldproben verfügbar waren. 
Die Messungen erfolgten im \textbf{Constant Current Mode}, mit den folgenden Parametern: 
\begin{align*}
    P &= 1000 \\
    I &= 2000 \\
    D &= 0
\end{align*}

Der Differentialregler $D$ wurde deaktiviert, da er äußere Störungen verstärken könnte. Zunächst wurden Aufnahmen eines $200 \, \text{nm} \times 200 \, \text{nm}$-Bereichs erstellt. Die z-Achse, welche die Oberflächenhöhe darstellt, wurde mithilfe einer Farbcodierung visualisiert. Zur besseren Übersicht wurde die Farbskala invertiert. In Abb. \ref{fig:gold-probe} sind die wolkenartigen Strukturen der Goldoberfläche gut erkennbar. Das nahezu fehlerfreie Bild zeigt, dass die gewählten STM-Einstellungen gut funktionierten.

\begin{figure}[h!]
    \centering
    \includegraphics[width=0.45\textwidth]{figure/Gold_mess_2.jpg}
    \hfill
    \includegraphics[width=0.45\textwidth]{figure/Gold_mess_inv.jpg}
    \caption{STM-Aufnahmen eines $200 \, \text{nm} \times 200 \, \text{nm}$-Bereichs der Goldprobe. Links: Originalbild, rechts: invertierte Farbskala. Points per line=$128$, Rastergeschwindigkeit = $ \SI{1}{\um \per \s}$ , Tip  Voltage = $750mV$, Setpoint = $1nA$}
    \label{fig:gold-probe}
\end{figure}
Anschließend wurde ein kleinerer Bereich von $20 \, \text{nm}$ aufgenommen (Abb. \ref{fig:gold-probeb}). Die Darstellung zeigt mehr Imperfektionen, vermutlich aufgrund von Verunreinigungen der Probe oder der Spitze. Es ist auch möglich, dass die Regelparameter für diesen kleineren Bereich nicht optimal eingestellt waren. Trotz der Störungen sind Elektronenwolken weiterhin erkennbar.

\begin{figure}[h!]
    \centering
    \includegraphics[width=0.45\textwidth]{figure/Gold_mess_3.jpg}
    \hfill
    \includegraphics[width=0.45\textwidth]{figure/Gold_mess_invb.jpg}
    \caption{STM-Aufnahmen eines $50 \, \text{nm} \times 50 \, \text{nm}$-Bereichs der Goldprobe. Links: Originalbild, rechts: invertierte Farbskala. Points per line=$128$, Rastergeschwindigkeit = $ \SI{0.4}{\um \per \s}$ , Tip  Voltage = $750mV$, Setpoint = $1nA$}
    \label{fig:gold-probeb}
\end{figure}

Die Körnigkeit der Oberflächenstruktur wurde mithilfe der Messfunktion in \textbf{Gwyddion} analysiert. Für zwei klar erkennbare Strukturen wurden folgende Durchmesser ermittelt:
\begin{align*}
    l_1 &= 23.37 \, \text{nm}, \\
    l_2 &= 32.90 \, \text{nm}.
\end{align*}
Diese Messungen sind jedoch als grobe Abschätzungen zu verstehen, da mögliche Verunreinigungen, beispielsweise durch Staubpartikel, die Ergebnisse verfälschen können. Staubpartikel in der Größenordnung von $>0.1 \, \mu\text{m}$ wären hier ungewöhnlich, jedoch könnten Ultrafeinstaubpartikel ($<0.1 \, \mu\text{m}$) in die beobachtete Größenordnung passen \cite{feinstaub}.

Abschließend fällt auf, dass sich die Elektronenwolken in den Abbildungen übereinander stapeln. Dies könnte durch statistische Effekte der Elektronenverteilung erklärt werden und sollte in zukünftigen Untersuchungen näher betrachtet werden.

\subsubsection*{Vergleich mit Anleitung}

Beim Vergleich unserer Ergebnisse (Abbildungen \ref*{fig:gold-probe} und \ref*{fig:gold-probeb}) mit denen 
aus der Anleitung \cite{Anleitung} zeigt sich, dass die Probe eher wie Gold auf Saphir als auf Silizium erscheint. 
Dieser Bildfehler könnte auf die verwendete Spitze zurückzuführen sein. Für diese Messung wurde eine ältere 
Spitze genutzt, die die Ergebnisse erheblich beeinflusst haben könnte. Zudem kam die Spitze zunächst in Kontakt 
mit dem Gold, bevor die Funktion "Approach" aktiviert wurde, was weitere Störungen in den Messungen verursacht 
haben könnte.


\begin{figure}[h!]
    \centering
    \includegraphics[width=.45\linewidth]{Gold_vergleich}
    \caption{Proben werden gezeigt. links: Gold auf Silizium(das Beobachtete), rechts: Gold auf Saphir}
    \label{fig:goldkugel_vergleich}
  \end{figure}



\newpage

\section{HOPG-Probe}
Als nächstes soll eine der HOPG-Proben (wir verwendeten Nummer 17) beobachtet werden.
Bei HOPG kann atomare Auflösung erreicht werden, weshalb dieses Material benutzt wird, um die Piezos, welche die STM-Spitze
bewegen, zu kalibrieren. Hier soll die Kalibrierung überprüft werden, indem Abbildungen gemacht werden und die daraus
bestimmte Gitterkonstante mit dem Literaturwert verglichen wird.

\subsection{Struktur von HOPG}
Ein Schema der Struktur von HOPG ist in Abb. \ref{fig:hopg-structure} gezeigt.
\begin{figure}[h]
    \centering
    \includegraphics[width=0.5\linewidth]{hopg-structure}
    \caption{
        Schema der Struktur von HOPG inklusive charakteristischen Längen \cite{graphite}. $a$ ist die Gitterkonstante.
        Die Blickrichtung mit dem STM liegt hier parallel zur $c$-Achse.
    }
    \label{fig:hopg-structure}
\end{figure}
Bei Betrachtung der Schichten senkrecht von oben, wie hier mit dem STM, gibt es drei relevante Formen sichtbarer Strukturen:
\begin{enumerate}
  \item Ein Atom in der obersten Schicht und keins in der Schicht darunter
  \item Ein Atom in der obersten Schicht und ein Atom in der Schicht darunter
  \item Kein Atom in der obersten Schicht, aber ein Atom in der Schicht darunter
\end{enumerate}
Diese drei Strukturen führen an der Oberfläche zu drei unterschiedlich hohen Elektronendichten,
die im STM-Bild in Form von drei verschiedenen Helligkeitsstufen (Hell, Halb-Hell, Dunkel) sichtbar werden.
In einem optimalen STM-Bild sollten z.B. die dunklen Punkte in
einem Gitter gleichseitiger Dreiecke mit Kantenlänge $a$ (Gitterkonstante; siehe Abb. \ref{fig:hopg-structure}) angeordnet sein.

\subsection{Messung}
Als nächstes soll eine der HOPG-Proben (wir verwendeten Nummer 17) beobachtet werden.
Bei HOPG kann atomare Auflösung erreicht werden, weshalb dieses Material benutzt wird, um die Piezos, welche die STM-Spitze
bewegen, zu kalibrieren. Hier soll die Kalibrierung überprüft werden, indem Abbildungen gemacht werden und die daraus
bestimmte Gitterkonstante mit dem Literaturwert verglichen wird.

\subsection{Struktur von HOPG}
Ein Schema der Struktur von HOPG ist in Abb. \ref{fig:hopg-structure} gezeigt.
\begin{figure}[h]
    \centering
    \includegraphics[width=0.5\linewidth]{hopg-structure}
    \caption{
        Schema der Struktur von HOPG inklusive charakteristischen Längen \cite{graphite}. $a$ ist die Gitterkonstante.
        Die Blickrichtung mit dem STM liegt hier parallel zur $c$-Achse.
    }
    \label{fig:hopg-structure}
\end{figure}
Bei Betrachtung der Schichten senkrecht von oben, wie hier mit dem STM, gibt es drei relevante Formen sichtbarer Strukturen:
\begin{enumerate}
  \item Ein Atom in der obersten Schicht und keins in der Schicht darunter
  \item Ein Atom in der obersten Schicht und ein Atom in der Schicht darunter
  \item Kein Atom in der obersten Schicht, aber ein Atom in der Schicht darunter
\end{enumerate}
Diese drei Strukturen führen an der Oberfläche zu drei unterschiedlich hohen Elektronendichten,
die im STM-Bild in Form von drei verschiedenen Helligkeitsstufen (Hell, Halb-Hell, Dunkel) sichtbar werden.
In einem optimalen STM-Bild sollten z.B. die dunklen Punkte in
einem Gitter gleichseitiger Dreiecke mit Kantenlänge $a$ (Gitterkonstante; siehe Abb. \ref{fig:hopg-structure}) angeordnet sein.

\subsection{Messung}
Optische Bilder der verwendeten Spitze und des TEM-Netzes sind in Abb. \ref{fig:spitze-2} und \ref{fig:gitter-hopg} gezeigt.
\begin{figure}[h]
    \centering
    \begin{minipage}{0.49\textwidth}
        \centering
        \includegraphics[width=\textwidth]{Spitze2}
        \includegraphics[width=\textwidth]{Spitze2}
        \caption{STM-Spitze, die zur Vermessung der HOPG-Probe verwendet wurde.
        Der Maßstab ergibt sich aus der Dicke des Drahts, welche \SI{0.3}{\mm} beträgt.
        Die Spitze wurde für diese Messung neu gerissen.}
        \label{fig:spitze-2}
    \end{minipage}
    \hfill
    \begin{minipage}{0.49\textwidth}
        \centering
        \includegraphics[width=\textwidth]{GitterHOPG}
        \includegraphics[width=\textwidth]{GitterHOPG}
        \caption{TEM-Netz zur Bestimmung des Maßstabs.}
        \label{fig:gitter-hopg}
    \end{minipage}
\end{figure}

Graphit besteht aus Schichten, welche untereinander nur schwach durch Van-der-Waals-Bindungen
zusammengehalten werden. Dadurch kann eine obere Schicht mithilfe eines Stücks Klebeband abgezogen werden.
Dies tun wir hier, um eine möglichst reine Oberfläche zu erhalten. Die Probe vor dem
zusammengehalten werden. Dadurch kann eine obere Schicht mithilfe eines Stücks Klebeband abgezogen werden.
Dies tun wir hier, um eine möglichst reine Oberfläche zu erhalten. Die Probe vor dem
Abziehen ist in Abb. \ref{fig:hopg-raw-lupe} und nach mehrfachem Abziehen in Abb. \ref{fig:hopg-nach-lupe}
gezeigt. Im optischen Bild ist kein großer Unterschied zu erkennen, da nur wenige Atomschichten entfernt wurden.

\begin{figure}[h]
    \centering
    \begin{minipage}{0.52\textwidth}
        \centering
        \includegraphics[width=\textwidth]{HOPG_roh}
        \includegraphics[width=\textwidth]{HOPG_roh}
        \caption{Verwendete HOPG-Probe vor dem Abziehen mit Klebeband.}
        \label{fig:hopg-raw-lupe}
    \end{minipage}
    \hfill
    \begin{minipage}{0.46\textwidth}
        \centering
        \includegraphics[width=\textwidth]{HOPGnach}
        \includegraphics[width=\textwidth]{HOPGnach}
        \caption{Verwendete HOPG-Probe nach dem Abziehen mit Klebeband.}
        \label{fig:hopg-nach-lupe}
    \end{minipage}
\end{figure}

Zunächst wird ein Testlauf im Constant Current Modus durchgeführt, um sicherzugehen, dass
es im betrachteten Ausschnitt keine größeren Stufen gibt.
Anschließend wird zum Constant Height Modus gewechselt, indem der P-Parameter auf $0$ und der
I-Parameter auf einen sehr niedrigen Wert ($4 - 10$) gesetzt werden.

Für die erste Abbildung wird mit einem verglichen zur Atomgröße relativ großen Bildausschnitt von $\SI{12.5}{\nm}$ begonnen (Abb. \ref{fig:hopg-large}). 
\begin{figure}[h]
Für die erste Abbildung wird mit einem verglichen zur Atomgröße relativ großen Bildausschnitt von $\SI{12.5}{\nm}$ begonnen (Abb. \ref{fig:hopg-large}). 
\begin{figure}[h]
    \centering
    \includegraphics[width=0.5\linewidth]{figure/hopg-large.jpg}
    \caption{STM-Bild von HOPG (Tip Current). Bildgröße \SI{12.5}{\nm}, $256$ Points per Line,
    $v=\SI{242.7}{\nm\per\s}$, Tip voltage=\SI{150}{\mV}, Setpoint=\SI{1}{\nA}, $I=10$, Overscan=$5\%$}
    \includegraphics[width=0.5\linewidth]{figure/hopg-large.jpg}
    \caption{STM-Bild von HOPG (Tip Current). Bildgröße \SI{12.5}{\nm}, $256$ Points per Line,
    $v=\SI{242.7}{\nm\per\s}$, Tip voltage=\SI{150}{\mV}, Setpoint=\SI{1}{\nA}, $I=10$, Overscan=$5\%$}
    \label{fig:hopg-large}
\end{figure}

Nach Anpassung einiger Parameter gelingt es schließlich auch, auf kleineren Skalen gute Bilder zu erhalten, siehe Abb. \ref{fig:hopg-medium} und \ref{fig:hopg-small}
\begin{figure}
    \centering
    \includegraphics[width=0.5\linewidth]{figure/hopg-medium.jpg}
    \includegraphics[width=0.5\linewidth]{figure/hopg-medium.jpg}
    \caption{
        STM-Bild von HOPG (Tip Current). Bildgröße \SI{3.125}{\nm}, $128$ Points per Line,
        STM-Bild von HOPG (Tip Current). Bildgröße \SI{3.125}{\nm}, $128$ Points per Line,
        $v=\SI{121.6}{\nm\per\s}$, Tip voltage=\SI{50.4}{\mV}, Setpoint=\SI{1}{\nA}, $I=4$, Overscan=$20\%$
    }
    \label{fig:hopg-medium}
\end{figure}

\begin{figure}[h]
    \centering
    \includegraphics[width=0.5\linewidth]{hopg-small}
    % \caption{
    %     STM-Bild von HOPG. Bildgröße \SI{3.125}{\nm}, $128$ Points per Line,
    %     $v=\SI{0.950}{\nm\per\s}$, Tip voltage=\SI{50.4}{\mV}, Setpoint=\SI{1}{\nA}, $I=4$
    % }
    \includegraphics[width=0.5\linewidth]{hopg-small}
    % \caption{
    %     STM-Bild von HOPG. Bildgröße \SI{3.125}{\nm}, $128$ Points per Line,
    %     $v=\SI{0.950}{\nm\per\s}$, Tip voltage=\SI{50.4}{\mV}, Setpoint=\SI{1}{\nA}, $I=4$
    % }
    \caption{
        STM-Bild von HOPG (Tip Current). Bildgröße \SI{1.56}{\nm}, $128$ Points per Line,
        STM-Bild von HOPG (Tip Current). Bildgröße \SI{1.56}{\nm}, $128$ Points per Line,
        $v=\SI{60.7}{\nm\per\s}$, Tip voltage=\SI{50.4}{\mV}, Setpoint=\SI{1}{\nA}, $I=4$, Overscan=$20\%$
    }
    \label{fig:hopg-small}
\end{figure}


\subsection{Auswertung}
Auch in dem größten Bild Abb. \ref{fig:hopg-large} lassen sich bereits regelmäßig angeordnete dunkle Stellen erkennen.
Diese sind jedoch nicht präzise genug aufgelöst, um genauere Messungen zu machen.

Im kleinsten Bild Abb. \ref{fig:hopg-small} sind die Strukturen deutlich größer zu sehen, zeigen jedoch eine deutliche Verzerrung.
Diese ist insbesondere am linken Bildrand stark ausgeprägt, also lässt sie sich auf Effekte beim Umdrehen der Bewegungsrichtung der
Spitze am Ende einer Linie zurückführen. Dies könnte prinzipiell vermutlich durch einen noch höher gewählten Overscan-Parameter
gelöst werden.

In Abb. \ref{fig:hopg-medium} hingegen sind über einen relativ großen Bereich regelmäßige Strukturen zu erkennen.
Insbesondere zum rechten Bildrand hin gibt es wieder Verzerrungen durch das Umkehren der Spitze, außerdem ist
jedoch eine vertikale Verzerrung in der oberen Bildhälfte sichbar. Diese lässt sich vermutlich durch thermische Drift erklären:
Temperaturänderungen der Probe (z.B durch einen Luftzug oder durch Aufheizen der STM-Elektronik) führen zu einer leichten Ausdehnung.
Findet diese Ausdehnung während der Scan-Bewegung der Spitze statt, wird der gleiche Punkt an der Oberfläche mehrmals aufgenommen
und es kommt zu einer Streckung des Bildes.

Im Zentrum des Bildes gibt es jedoch einen Bereich mit wenig Verzerrung.
Hier wurden mehrere Linien eingezeichnet, um aus dem Abstand der dunklen Punkte die Gitterkonstante $a$
zu bestimmen. Dabei wurden möglichst viele Gitterabstände überspannt, um den Fehler zu minimieren.
Die Maße der Linien sind in Tab. \ref{tab:hopg-data} eingetragen. $a$ ergibt sich durch $a=\frac{R}{n}$, wobei $n$
der Anzahl der von der Linie überspannten Gitterabständen entspricht;
$\Delta a$ wurde mithilfe von Gauß'scher Fehlerfortpflanzung berechnet.

\subsection{Auswertung}
Auch in dem größten Bild Abb. \ref{fig:hopg-large} lassen sich bereits regelmäßig angeordnete dunkle Stellen erkennen.
Diese sind jedoch nicht präzise genug aufgelöst, um genauere Messungen zu machen.

Im kleinsten Bild Abb. \ref{fig:hopg-small} sind die Strukturen deutlich größer zu sehen, zeigen jedoch eine deutliche Verzerrung.
Diese ist insbesondere am linken Bildrand stark ausgeprägt, also lässt sie sich auf Effekte beim Umdrehen der Bewegungsrichtung der
Spitze am Ende einer Linie zurückführen. Dies könnte prinzipiell vermutlich durch einen noch höher gewählten Overscan-Parameter
gelöst werden.

In Abb. \ref{fig:hopg-medium} hingegen sind über einen relativ großen Bereich regelmäßige Strukturen zu erkennen.
Insbesondere zum rechten Bildrand hin gibt es wieder Verzerrungen durch das Umkehren der Spitze, außerdem ist
jedoch eine vertikale Verzerrung in der oberen Bildhälfte sichbar. Diese lässt sich vermutlich durch thermische Drift erklären:
Temperaturänderungen der Probe (z.B durch einen Luftzug oder durch Aufheizen der STM-Elektronik) führen zu einer leichten Ausdehnung.
Findet diese Ausdehnung während der Scan-Bewegung der Spitze statt, wird der gleiche Punkt an der Oberfläche mehrmals aufgenommen
und es kommt zu einer Streckung des Bildes.

Im Zentrum des Bildes gibt es jedoch einen Bereich mit wenig Verzerrung.
Hier wurden mehrere Linien eingezeichnet, um aus dem Abstand der dunklen Punkte die Gitterkonstante $a$
zu bestimmen. Dabei wurden möglichst viele Gitterabstände überspannt, um den Fehler zu minimieren.
Die Maße der Linien sind in Tab. \ref{tab:hopg-data} eingetragen. $a$ ergibt sich durch $a=\frac{R}{n}$, wobei $n$
der Anzahl der von der Linie überspannten Gitterabständen entspricht;
$\Delta a$ wurde mithilfe von Gauß'scher Fehlerfortpflanzung berechnet.

\begin{table}[h]
    \centering
    \begin{tabular}{c||c|c|c|c|c}
        Linie Nr. & $\phi/\si{\degree}$ & $R/\si{\nm}$ & $n$ & $a/\si{\nm}$ & $\Delta a/\si{\nm}$ \\
        \hline
        $1$ & $17.0 $   & $1.280$ & $5$ & $0.256$ & $0.010$ \\
        $2$ & $-49.0$	& $0.642$ & $3$ & $0.214$ & $0.017$ \\
        $3$ & $16.4	$   & $1.279$ & $5$ & $0.256$ & $0.010$ \\
        $4$ & $-50.9$	& $0.636$ & $3$ & $0.212$ & $0.017$
    \end{tabular}
    \caption{gemessene Längen und Winkel im HOPG. $\Delta\phi=\ang{3.0}$, $\Delta R = \SI{0.050}{\nm}$. $\phi$ ist der Winkel der Linie zur Horizontalen, R die Länge der Linie und $n$ steht für die Anzahl an Gitterabständen, welche die Linie überspannt.
    Die Nummerierung der Linien kommt aus Abb. \ref{fig:hopg-medium} (1 und 3 sind die längeren, 2 und 4 die kürzeren Linien).}
    \label{tab:hopg-data}
\end{table}

Der Literaturwert ist $a=\SI{0.246}{\nm}$, wovon der aus Linien $1$ und $3$ bestimmte $a$-Wert um ca. $\SI{4}{\percent}$
nach oben abweicht. Die Werte aus Linien $2$ und $4$ hingegen sind um mehr als \SI{10}{\percent} niedriger.
Dies deutet darauf hin, dass die Piezos für die x- und y- Richtung nicht optimal kalibriert sind.
Linien $1$ und $3$, die bessere $a$-Werte liefern, verlaufen vorwiegend horizontal, also scheint die Kalibration des x-Piezos
eher zu stimmen. Linien $2$ und $4$, die eine größere Ausdehnung in der Vertikalen haben, sind deutlich zu kurz,
also muss der $y-Piezo$ so kalibriert sein, dass er sich zu weit bewegt und somit Abstände unterschätzt werden.
Der varianzgewichtete Mittelwert der Gitterkonstanten aus Tab. \ref{tab:hopg-data} ist $a = \SI{0.244\pm0.012}{\nm}$,
was innerhalb des Fehlerbereichs mit dem Literaturwert übereinstimmt und um weniger als $1\%$ von ihm abweicht.

Dass die Kalibration in y-Richtung stärker abweicht, als in x-Richtung,
führt zu einer Verzerrung des Bilds, welche sich auch in den Winkeln zwischen Linien erkennen lässt: Die Differenz der
Winkel $\phi$ der Linien $1$ und $2$ ist \SI{66.0\pm 4.2}{\degree} und zwischen Linien $3$ und $4$ \SI{67.3\pm 4.2}{\degree},
was um ca. \SI{10}{\percent} höher ist, als der Winkel \SI{66}{\degree}, der bei einem gleichseitigen Dreieck eigentlich zu
erwarten wäre.
% Aus Linien $1$ und $3$ erhalten wir $a=0.256$ und
Der Literaturwert ist $a=\SI{0.246}{\nm}$, wovon der aus Linien $1$ und $3$ bestimmte $a$-Wert um ca. $\SI{4}{\percent}$
nach oben abweicht. Die Werte aus Linien $2$ und $4$ hingegen sind um mehr als \SI{10}{\percent} niedriger.
Dies deutet darauf hin, dass die Piezos für die x- und y- Richtung nicht optimal kalibriert sind.
Linien $1$ und $3$, die bessere $a$-Werte liefern, verlaufen vorwiegend horizontal, also scheint die Kalibration des x-Piezos
eher zu stimmen. Linien $2$ und $4$, die eine größere Ausdehnung in der Vertikalen haben, sind deutlich zu kurz,
also muss der $y-Piezo$ so kalibriert sein, dass er sich zu weit bewegt und somit Abstände unterschätzt werden.
Der varianzgewichtete Mittelwert der Gitterkonstanten aus Tab. \ref{tab:hopg-data} ist $a = \SI{0.244\pm0.012}{\nm}$,
was innerhalb des Fehlerbereichs mit dem Literaturwert übereinstimmt und um weniger als $1\%$ von ihm abweicht.

Dass die Kalibration in y-Richtung stärker abweicht, als in x-Richtung,
führt zu einer Verzerrung des Bilds, welche sich auch in den Winkeln zwischen Linien erkennen lässt: Die Differenz der
Winkel $\phi$ der Linien $1$ und $2$ ist \SI{66.0\pm 4.2}{\degree} und zwischen Linien $3$ und $4$ \SI{67.3\pm 4.2}{\degree},
was um ca. \SI{10}{\percent} höher ist, als der Winkel \SI{66}{\degree}, der bei einem gleichseitigen Dreieck eigentlich zu
erwarten wäre.
% Aus Linien $1$ und $3$ erhalten wir $a=0.256$ und


\clearpage
\section{Fazit}
Die Funktionsweise und den Einsatz eines Rastertunnelmikroskops (RTM) wurde während des Versuchs erfolgreich untersucht. Es wurden Goldproben analysiert, wobei die erwarteten Aufnahmen erzielt wurden, die den theoretischen Vorhersagen entsprachen. Die Oberflächenbehandlung der Proben führte zur Beobachtung wolkenähnlicher Strukturen, wie prognostiziert. 

Außerdem wurden mögliche Ursachen für Bildfehler untersucht, darunter die Präparation der Probe, die Empfindlichkeit der Geräte und äußere Störeinflüsse. Diese Analyse lieferte wertvolle Einblicke in die Herausforderungen und die Präzisionsanforderungen bei der Bildgebung mit dem RTM.







\clearpage
\section{Anhang}
Im Folgenden sind weitere Bilder der Proben gezeigt, die aufgenommen wurden aber hier nicht weiter analysiert werden.

\subsection{HOPG-Probe}
Siehe Abb. \ref{fig:hopg-1}, \ref{fig:hopg-2}, \ref{fig:hopg-3}. Da im Constant Height-Modus gearbeitet wird, gilt immer $P=0$ und $D=0$.

\begin{figure}[h]
    \centering
    \begin{minipage}{0.49\textwidth}
        \centering
        \includegraphics[width=\textwidth]{Image20212}
        \caption{
            STM-Bild von HOPG (Tip Current). Bildgröße \SI{3.125}{\nm}, $512$ Points per Line,
            $v=\SI{20.7}{\nm\per\s}$, Tip voltage=\SI{150}{\mV}, Setpoint=\SI{1}{\nA}, $I=10$, Overscan=$5\%$
        }
        \label{fig:hopg-1}
    \end{minipage}
    \hfill
    \begin{minipage}{0.49\textwidth}
        \centering
        \includegraphics[width=\textwidth]{Image20215}
        \caption{
            STM-Bild von HOPG (Tip Current). Bildgröße \SI{3.125}{\nm}, $128$ Points per Line,
            $v=\SI{121.6}{\nm\per\s}$, Tip voltage=\SI{50.4}{\mV}, Setpoint=\SI{1}{\nA}, $I=4$, Overscan=$10\%$
        }
        \label{fig:hopg-2}
    \end{minipage}

    \begin{minipage}{0.49\textwidth}
        \centering
        \includegraphics[width=\textwidth]{Image20221.jpg}
        \caption{
            STM-Bild von HOPG (Tip Current). Bildgröße \SI{1.56}{\nm}, $128$ Points per Line,
            $v=\SI{60.8}{\nm\per\s}$, Tip voltage=\SI{50.4}{\mV}, Setpoint=\SI{1}{\nA}, $I=4$, Overscan=$20\%$
        }
        \label{fig:hopg-3}
    \end{minipage}
\end{figure}



\clearpage
\begin{thebibliography}{9}

\bibitem{Anleitung}
\textit{Physikalisches Praktikum Teil IV -- Versuchsbeschreibungen}, Universität Bonn, 10.10.2024

\bibitem{stm-paper} 
\textit{High-Speed Electrochemical STM}, M.J. Rost, 2018

\bibitem{naio-video}
\textit{NaioSTM Overview and Tutorial}, nanosurf, Abruf 27.11.2024, https://youtu.be/dZeZn3Pt2Ss
\bibitem{stm-paper} 
\textit{High-Speed Electrochemical STM}, M.J. Rost, 2018

\bibitem{naio-video}
\textit{NaioSTM Overview and Tutorial}, nanosurf, Abruf 27.11.2024, https://youtu.be/dZeZn3Pt2Ss

\bibitem{meshsize}
\textit{Was Sie über TEM-Grids wissen sollten}, Science Services, Abruf 29.11.2024

\bibitem{graphite}
\textit{Graphite}, mindat.org, Abruf 29.11.2024, https://www.mindat.org/min-1740.html

\bibitem{feinstaub}
\textit{Feinstaub}, Chemie.de, Abruf 24.11.2024, \url{https://www.chemie.de/lexikon/Feinstaub.html}

\bibitem{easyScan}
\textit{Operating Instructions easyScan 2 STM}, Version 2.2, Januar 2010, PROD.NR.: BT02090-11, NANOSURF AG.


\end{thebibliography}

\end{document}

